\begin{frame}
	\frametitle{Introdução}
	
	\only<1>{
		\framesubtitle{Bibliografia recomendada}
		\begin{itemize}
			\item KLEINBERG, J., TARDOS, E.; Algorithm Design, Addison-Weslwy, 2005.
			\item CORMEN, T. H., LEISERSON, C. E., RIVEST, R. L., STEIN, C.;
			\item Introduction to	Algorithms, 3rd Edition, The MIT Press, 2009.
			\item SKIENA, S. S., The algorithm design manual, Springer, 2010.
		\end{itemize}
	}

	\only<2>{		
		\framesubtitle{Objetivos e ementa}
		\par Espera-se que ao final do semestre o aluno domine as técnicas para análise de algoritmos, envolvendo o comportamento assintótico de sua execução e de sua eficiência na solução de problemas específicos.
		Espera-se também, que o mesmo tenha condições de julgar e escolher as melhores alternativas, tanto do ponto de vista de tempo como espaço, para algoritmos aplicados em	problemas reais.
	}

	\only<3>{		
		\framesubtitle{Conteúdo}
		\begin{itemize}
			\item Revisão de álgebra (funções assintóticas).
			\item Relações de recorrência.
			\item Análise de algoritmos através de funções assintóticas, incluindo notação Big-O e demais funções assintóticas.
			\item Análise de problemas NP-completo e reducibilidade.
			\item Aplicação da análise assintótica na avaliação de algoritmos de ordenação.
			\item Análise e projeto de algoritmos de busca.
			\item Algoritmos randômicos.
			\item Algoritmos baseados em abordagem gulosa.
			\item Algoritmos baseados e programação dinâmica.
			\item Algoritmos para grafos: caminho mínimo, árvore geradora, detecção de ciclos, etc.
		\end{itemize}
	}

	\only<4>{		
		\framesubtitle{Avaliações}
		\par Provas práticas (análise e programação) e escritas (discussões).
	}
\end{frame}