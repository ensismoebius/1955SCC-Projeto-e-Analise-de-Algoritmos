
\begin{frame}
	\frametitle{Análise de problemas P / NP}
	\par Inicialmente definiremos as classes de problemas P/NP para problemas de \textbf{decisão}, ou seja, problemas cujas respostas são \textbf{sim/não}
	
	\par No entanto, \textbf{pode ser} relativamente fácil converter um problema de otimização em um de decisão.
	
	\par \textbf{Por exemplo}:
	\par \textit{Existe um caminho de distância $K$ entre dois pontos de um grafo?} A resposta será sim ou não. \newline
	\par De forma mais genérica: Existe uma solução que satisfaz uma certa propriedade?\newline
	\par \textbf{No entanto} podemos estimar um valor para $K$ e ir perguntando ao algoritmo que, caso responda \textbf{não}, poderá responder sim para $K + j$ ou $K -j$ tal que $j$ seja um valor que faça com que exista uma solução. A forma de encontrar esse $j$ pode variar de acordo com a situação. Ou seja, associado a um problema de decisão pode estar vinculado um problema de \textbf{otimização}.
\end{frame}

\begin{frame}
	\frametitle{Análise de problemas P / NP}
	\framesubtitle{Algoritmos polinomiais}
	\par Algoritmos polinomiais são aqueles que podem ter seu \textbf{tempo representado na forma de um polinômio} de grau qualquer, basicamente quase todos os que vimos até agora,  por outro lado, algoritmos não polinomiais são aqueles cuja a representação é dada, por exemplo, por funções exponenciais.\newline
	\par A verdade é que não se pode ter \textbf{total certeza} que um algoritmo é não polinomial pois pode ser que, para um determinado problema, exista uma solução polinomial portanto, o que se pode afirmar é que \textbf{até o momento} se sabe que a solução desses problemas tem algoritmos de tempo não polinomial. 
\end{frame}

\begin{frame}
	\frametitle{Análise de problema P / NP}
	\framesubtitle{Algoritmos polinomiais}
	\par Dizemos que um algoritmo resolve um dado problema se, ao receber uma entrada do problema, devolve uma solução ou informa que não há solução.
	\par Se diz que um algoritmo é \textbf{razoavelmente rápido} quando o mesmo \textbf{tem um tempo polinomial}.
	\par Um algoritmo é polinomial se \textbf{o pior caso} do mesmo for algo como $O(n^i) \forall i \in  \mathbb{N}$
	\par Exemplos:
	\begin{itemize}
		\item $10n^3+3n^2+1 \implies O(n^3)$
		\item $n^5.\log n \leq n^6 \implies o(n^6)$
	\end{itemize}
	\par É importante separar os algoritmos polinomiais dos não polinomiais pois polinômios tem algumas características interessantes como fechamento em \textbf{adição, multiplicação e composição} que, quando aplicadas \textbf{não mudam} a natureza polinomial do tempo do algoritmo.
\end{frame}

\begin{frame}
	\frametitle{Análise de problemas P / NP}
	\framesubtitle{Algoritmos \textbf{não} polinomiais}
	\par São todos aqueles que consomem um tempo \textbf{não polinomial} como, por exemplo, $e^n$, $n!$, $2^n$, etc.
	\par Geralmente a resolução desses problemas é custosa e se dá pela estratégia da \textbf{força bruta}. É verdade que, ainda assim, é possível economizar alguns ciclos de processamento usando uma \textbf{árvore de estados} mas isso não muda o tempo assintótico do algoritmo.
\end{frame}

\begin{frame}
	\frametitle{Análise de problemas P / NP}
	\framesubtitle{A classe P de problemas}
	\par A classe $P$ de problemas compreende todos os algoritmos que podem ser executados em um tempo polinomial, \textbf{o fato de ainda não se ter encontrado um algoritmo polinomial para um problema não garante \textit{definitivamente} que a natureza do problema seja não polinomial}.
\end{frame}

\begin{frame}
	\frametitle{Análise de problemas P / NP}
	\framesubtitle{O problema da satisfazibilidade  (S.A. Cook, L.A. Levin, 1973)}
	\par Dada uma fórmula booleana na forma normal \textbf{conjuntiva} ou \textbf{disjuntiva}. Existem valores para $x_1,x_2,x_3, \dots x_n$ que resultam em \textbf{Verdadeiro} para essa fórmula?
	\par Exemplos de instâncias:
	\begin{enumerate}
		\item conjuntiva: $(x_1 \vee \neg x_2)\wedge(x_1 \vee x_3)\wedge(\neg x_1 \vee x_2)$ \label{enum:conj}
		\item disjuntiva: $(x_1 \wedge \neg x_2)\vee(x_1 \wedge x_3)\vee(\neg x_1 \wedge x_2)$\label{enum:disjun}
	\end{enumerate}
	\par À \textbf{solução} dessa fórmula daremos o nome de \textbf{certificado}.
	\par O \textbf{certificado} garante que, dado o problema, teremos a resposta \textbf{sim}. De outra forma: O certificado é a \textbf{prova} de que a resposta sim é verdadeira.
	\par Para o exemplo \ref{enum:conj} o certificado é $x_1=Verdadeiro, x_2=Verdadeiro$\newline
	
	\par \textbf{Pergunta}: Existe um \textbf{certificado} para o problema \ref{enum:disjun}? Qual?
	
	\par Considerando o problema da satisfazibilidade entremos no campo dos problemas \textbf{NP}. Mais especificamente \textbf{NP-completo}.
\end{frame}

\begin{frame}
	\frametitle{Análise de problemas P / NP}
	\framesubtitle{A classe NP de problemas}
	\par A classe $NP$ é aquela dos algoritmos não polinomiais certo? \textbf{Errado!}
	\par \textbf{NP} é um acrônimo para \textit{Non-deterministic polynomial time}, ou seja, é uma classe de problemas cuja validade de um certificado pode ser avaliada em tempo \textbf{polinomial} e cuja a solução pode ser obtida em tempo polinomial usando um sistema de computação \textbf{hipotético} não determinístico (o que se aproxima mais disso são os atuais computadores quânticos).
	\par De outra forma: Dada uma solução hipotética para um problema NP é rápido e fácil (tempo polinomial) verificar se aquela solução é \textbf{verdadeira}. Já, descobrir uma solução para o problema, bom... aí é outra estória pois esse tipo de problema \textbf{não tem algoritmos conhecidos} cujo tempo de geração de um certificado seja polinomial.
	\par Daí tiramos que $P \subset NP$.
	\par Obs: \textbf{Não se sabe} se $P = NP$ ou não, quem conseguir provar isso (que $P = NP$ ou  $P \neq NP$) revolucionará a área da computação.	
\end{frame}

\begin{frame}
	\frametitle{Análise de problemas P / NP}
	\framesubtitle{A classe NP de problemas}
	\par Considerando que qualquer problema que possa, de alguma forma, ser transformado no problema da satisfazibilidade ou em qualquer outro problema NP \textbf{também é NP}, afim de provar que $P=NP$ ou $P \neq NP$ basta provar isso para \textbf{um} único problema $NP$!
	\newline
	\par A página do professor \textit{Paulo Feofiloff} tem muitos exemplos:
	\href{https://www.ime.usp.br/~pf/analise_de_algoritmos/aulas/NPcompleto.html}{https://www.ime.usp.br/\texttildelow pf/analise\_de\_algoritmos/aulas/NPcompleto.html} 
\end{frame}

\begin{frame}
	\frametitle{Análise de problemas P / NP}
	\framesubtitle{Redução polinomial}
	\par Da definição anterior tiramos que: Seja $A$ um problema e $B$ outro cuja solução é conhecida. Se a conversão de $A$ para $B$ é possível então solucionar $A$ também o será.
	\par Ao procedimento descrito acima damos o nome de redução polinomial de $A$ para $B$ ou $A \preceq B$ (lê-se A é redutível a B) onde $B$ é um algoritmo polinomial
	\par É importante dizer que $B\subseteq A$ e $B$ pode ser chamado múltiplas vezes em $A$.\newline
	
	\par \textbf{Exemplo}: \textit{Selecione o enésimo termo de uma sequência numérica randômica.}
	\par Tal problema pode ser reduzido a uma de ordenação seguida de uns poucos comandos.	
	
	\par Sendo assim, suponha que $A \preceq B$ então:
	\begin{itemize}
		\item se $B \in P \implies A \in P$
		\item se $B \preceq C \implies A \preceq C$
		\item se $B \in P \implies A \in P$
		\item se $B \in NP$ \textbf{talvez} haja uma solução de $A \in P$ que não use $B$.
		\item se $A$ não tem solução polinomial e $A \preceq B / B \in NP \implies A \in NP$
	\end{itemize}
\end{frame}

\begin{frame}
	\frametitle{Análise de problemas P / NP}
	\framesubtitle{Redução polinomial - NP-completo}
	\par \textbf{Então} se $\mathbf{NP-completo} \subset NP, A \in NP, B \preceq A \forall B \in NP \implies A \in \mathbf{NP-completo}$\newline
	\par Embora NP (e NP-completo) de tenha como domínio os problemas de decisão, \textbf{como ja vimos} existe uma correlação entre problemas de otimização e de decisão que garante que o problema de decisão correspondente ao de otimização deve ser pelo menos \textbf{não mais difícil} que o de otimização. 
	\par De outra forma: Se um problema de otimização é polinomial sua respectiva decisão também o será, \textbf{o inverso não é necessariamente verdade}.
	
	\par \textbf{Finalmente} problemas NP-completos são aqueles que podem de alguma forma serem reduzidos ao problema da satisfazibilidade (meh...)
\end{frame}

\begin{frame}
	\frametitle{Análise de problemas P / NP}
	\framesubtitle{Exercício 0}
	\par Crie um programa que:
	\begin{itemize}
		\item verifique se um número é primo ou não
		\item gere o enésimo numero primo dado um valor de $n$
	\end{itemize}
	\par \textbf{Determine o tempo de execução de ambos}
	\par \textbf{Algum dos dois é NP?}
	\par \textbf{Qual a relação entre eles?}
\end{frame}

\begin{frame}
	\frametitle{Análise de problemas P / NP}
	\framesubtitle{Exercício 1}
	\begin{itemize}
		\item Determine um circuito hamiltoniano de custo mínimo.
		\item Dado um valor $L$ existe um circuito de custo $L$?
	\end{itemize}
	\par \textbf{Determine o tempo de execução de ambos}
	\par \textbf{Algum dos dois é NP?}
	\par \textbf{Qual a relação entre eles?}\newline
	
	\par Um \textit{circuito hamiltoniano} é um circuito/ciclo cujos elementos não se repetem.
\end{frame}

\begin{frame}
	\frametitle{Análise de problemas P / NP}
	\framesubtitle{Exercício 2}
	\begin{itemize}
		\item Determine quantas cliques com 2 e 3 elementos um certo grafo pode ter.
		\item Dado uma quantidade e elementos $k$ existe um clique com $k$ elementos?
	\end{itemize}
	\par \textbf{Determine o tempo de execução de ambos}
	\par \textbf{Algum dos dois é NP?}
	\par \textbf{Qual a relação entre eles?}\newline
	
	\par Uma \textit{clique} é um subgrafo completo.
\end{frame}

\begin{frame}
	\frametitle{Análise de problemas P / NP}
	\framesubtitle{Exercício 3}
	\begin{itemize}
		\item Dado um grafo o mesmo é euleriano?
	\end{itemize}
	\par \textbf{Determine o tempo de execução}
	\par \textbf{Esse algoritmo é NP?}\newline
	
	\par Um grafo euleriano é aquele em que é possível percorrer todas as arestas do mesmo podendo apenas repetir os vértices. Grafos eulerianos tem todos os vértices com grau par.
\end{frame}

\begin{frame}
	\frametitle{Análise de problemas P / NP}
	\framesubtitle{Exercício 4}
	\par Considere $S \subseteq \mathbb{Z} / k \in S, |S| = n, n \in \mathbb{N}$, ou seja, $S$ é subconjunto de $\mathbb{Z}$ e a quantidade de elementos de $S$ é igual a um $n$ qualquer natural, então: Existe $\sum_{i \in S} k_i = \sum_{i \notin S} k_i$?
\end{frame}
